\documentclass[12pt]{article}
\usepackage[left=0.25cm,top=1cm,right=0.25cm,bottom=1cm]{geometry}
%\usepackage[landscape]{geometry}
\textwidth = 20cm
\hoffset = -1cm
\usepackage[utf8]{inputenc}
\usepackage[spanish,es-tabla, es-lcroman]{babel}
\usepackage[autostyle,spanish=mexican]{csquotes}
\usepackage[tbtags]{amsmath}
\usepackage{nccmath}
\usepackage{amsthm}
\usepackage{amssymb}
\usepackage{mathrsfs}
\usepackage{graphicx}
\usepackage{subfig}
\usepackage{caption}
%\usepackage{subcaption}
\usepackage{standalone}
\usepackage[outdir=./Imagenes/]{epstopdf}
\usepackage{siunitx}
\usepackage{physics}
\usepackage{color}
\usepackage{float}
\usepackage{hyperref}
\usepackage{multicol}
\usepackage{multirow}
%\usepackage{milista}
\usepackage{anyfontsize}
\usepackage{anysize}
%\usepackage{enumerate}
\usepackage[shortlabels]{enumitem}
\usepackage{capt-of}
\usepackage{bm}
\usepackage{mdframed}
\usepackage{relsize}
\usepackage{placeins}
\usepackage{empheq}
\usepackage{cancel}
\usepackage{pdfpages}
\usepackage{wrapfig}
\usepackage[flushleft]{threeparttable}
\usepackage{makecell}
\usepackage{fancyhdr}
\usepackage{tikz}
\usepackage{bigints}
\usepackage{menukeys}
\usepackage{tcolorbox}
\tcbuselibrary{breakable}
\usepackage{scalerel}
\usepackage{pgfplots}
\usepackage{pdflscape}
\pgfplotsset{compat=1.16}
\spanishdecimal{.}
\renewcommand{\baselinestretch}{1.5} 
\renewcommand\labelenumii{\theenumi.{\arabic{enumii}})}

\newcommand{\python}{\texttt{python}}
\newcommand{\textoazul}[1]{\textcolor{blue}{#1}}
\newcommand{\azulfuerte}[1]{\textcolor{blue}{\textbf{#1}}}
\newcommand{\funcionazul}[1]{\textcolor{blue}{\textbf{\texttt{#1}}}}

\newcommand{\pderivada}[1]{\ensuremath{{#1}^{\prime}}}
\newcommand{\sderivada}[1]{\ensuremath{{#1}^{\prime \prime}}}
\newcommand{\tderivada}[1]{\ensuremath{{#1}^{\prime \prime \prime}}}
\newcommand{\nderivada}[2]{\ensuremath{{#1}^{(#2)}}}


\newtheorem{defi}{{\it Definición}}[section]
\newtheorem{teo}{{\it Teorema}}[section]
\newtheorem{ejemplo}{{\it Ejemplo}}[section]
\newtheorem{propiedad}{{\it Propiedad}}[section]
\newtheorem{lema}{{\it Lema}}[section]
\newtheorem{cor}{Corolario}
\newtheorem{ejer}{Ejercicio}[section]

\newlist{milista}{enumerate}{2}
\setlist[milista,1]{label=\arabic*)}
\setlist[milista,2]{label=\arabic{milistai}.\arabic*)}
\newlength{\depthofsumsign}
\setlength{\depthofsumsign}{\depthof{$\sum$}}
\newcommand{\nsum}[1][1.4]{% only for \displaystyle
    \mathop{%
        \raisebox
            {-#1\depthofsumsign+1\depthofsumsign}
            {\scalebox
                {#1}
                {$\displaystyle\sum$}%
            }
    }
}
\def\scaleint#1{\vcenter{\hbox{\scaleto[3ex]{\displaystyle\int}{#1}}}}
\def\scaleoint#1{\vcenter{\hbox{\scaleto[3ex]{\displaystyle\oint}{#1}}}}
\def\scaleiiint#1{\vcenter{\hbox{\scaleto[3ex]{\displaystyle\iiint}{#1}}}}
\def\bs{\mkern-12mu}

\newcommand{\Cancel}[2][black]{{\color{#1}\cancel{\color{black}#2}}}



\author{M. en C. Gustavo Contreras Mayén. \texttt{curso.fisica.comp@gmail.com}\\
M. en C. Abraham Lima Buendía. \texttt{abraham3081@ciencias.unam.mx}}
\title{Syllabus del Curso de Óptica \\ {\large Semestre 2024-2 Grupo 8365}}
\date{ }
% \makeatletter
% \renewcommand{\@biblabel}[1]{}
% \renewenvironment{thebibliography}[1]
%      {\section*{\refname}%
%       \@mkboth{\MakeUppercase\refname}{\MakeUppercase\refname}%
%       \list{}%
%            {\labelwidth=0pt
%             \labelsep=0pt
%             \leftmargin1.5em
%             \itemindent=-1.5em
%             \advance\leftmargin\labelsep
%             \@openbib@code
%             }%
%       \sloppy
%       \clubpenalty4000
%       \@clubpenalty \clubpenalty
%       \widowpenalty4000%
%       \sfcode`\.\@m}
% \makeatother

\usepackage[backend=biber, style=ieee, sorting=ynt]{biblatex}
\addbibresource{LibrosFC.bib}


\begin{document}

\renewcommand\labelenumii{\theenumi.{\arabic{enumii}}}
\maketitle
\fontsize{12}{12}\selectfont

\textbf{Lugar: } Pendiente.
\par
\textbf{Horario: } Lunes, Miércoles y Viernes de 18 a 20 pm.
\par
\par
\section{Objetivos y Temario:}

Se trabajará el temario oficial de la asignatura, que está disponible en:

\href{https://www.fciencias.unam.mx/sites/default/files/temario/715.pdf}{https://www.fciencias.unam.mx/sites/default/files/temario/715.pdf}

\section{Metodología de Enseñanza.}

\noindent
\textbf{Antes de la clase.}
\\
Para facilitar la discusión en el aula, el alumno revisará el material de trabajo que se le proporcionará oportunamente, así como la solución de algunos ejercicios, de tal manera que llegará a la clase conociendo el tema a desarrollar. Daremos por entendido de que el alumno realizará la lectura y actividades establecidas.
\\
\textbf{Durante la clase.}
\\
Se dará un tiempo para la exposición con diálogo y discusión del material de trabajo con los temas a cubrir durante el semestre. Se busca que sea un curso práctico por lo que se va a trabajar con los equipos de cómputo del laboratorio, de tal manera que se resolverán ejercicios en clase.
\par
Un curso de este tipo requiere que el alumno plantee un problema de la física y proponga una solución mediante un algoritmo computacional, de tal forma que va a requerir \enquote{ejecutar} su algoritmo para verificar la funcionalidad del mismo, así como revisar la congruencia de la solución.
\par
Considera que el curso no está enfocado al desarrollo de habilidades y/o técnicas de programación con un lenguaje en particular. En un primer momento se revisará un planteamiento general de la solución, para posteriormente, implementarlo en la sintaxis del lenguaje \python.
\par
Si cuentan con una experiencia previa en programación (con cualquier lenguaje), será conveniente para el trabajo en clase; pero si no han programado, se verán en la necesidad de dedicarle más tiempo tanto para revisar los materiales adicionales, así como para resolver los problemas y ejercicios.
\\
\textbf{Después de la clase.}
\\
Al concluir la clase, se tendrán ejercicios a resolver, para que pueda repasar el tema visto en clase. En caso de que algún ejercicio haya quedado incompleto, deberá de resolver y entregarlos en la plataforma Moodle.

\subsection{Plataforma Moodle.}

En este semestre el curso se impartirá en modalidad presencial, se mantendrá el uso de la plataforma Moodle para el mismo, en donde se incluirán materiales de consulta: ejercicios adicionales, lecturas, artículos, enlaces a videos, a archivos para utilizar, por lo que se tendrá un apoyo adicional, de manera que contarán con herramientas adicionales.
\par
Los ejercicios resueltos se dejarán en Moodle, contando con el suficiente tiempo para la entrega. No se recibirán actividades por correo electrónico.

\section{Evaluación.}

La evaluación de las actividades de un curso como el de Física Computacional debe de ser distinto a los cursos \enquote{teóricos}, ya que aunque se tenga la referencia sobre el objetivo o resultado, la manera en la que se resuelve un ejercicio puede ser distinta, la pregunta entonces es: ¿cómo se deben de evaluar actividades que reportan el mismo resultado?. Para favorecer en el alumno el desarrollo y la solución de los ejercicios, se contará con una \textbf{rúbrica} que lo guiará en los puntos que hay que cubrir en la entrega de un ejercicio resuelto. Esta guía le permitirá estimar al alumno sobre el cumplimiento en su solución, este tipo de estrategia se conoce como \textbf{autoevaluación}, además de recibir una revisión y evaluación por parte del equipo académico del curso.
\par
Los elementos y la proporción de la calificación total del curso, se distribuyen de la siguiente manera:
\begin{enumerate}[label=\alph*)]
\item \textbf{Ejercicios en clase $\mathbf{10\%}$:} Durante la clase se trabajarán ejercicios, algunos de ellos se dejarán para que completen la solución, de tal forma que deberán de entregarlo resuelto para la siguiente sesión.
\par
Para que el ejercicio resuelto se considere dentro de este porcentaje, se requiere que el alumno asista a la clase, en caso de que el alumno no asista y se entere del ejercicio, solamente se le revisará el ejercicio que entregue, pero no se le tomará en cuenta para el porcentaje, (moraleja: hay que asistir a clase).
\item \textbf{Examen-Tarea $\mathbf{50\%}$} : Se tendrán tres examen-tarea durante el curso, se les proporcionará de manera adelantada y con fecha de entrega definida, no se recibirán entregas extemporáneas. Para que el examen-tarea se considere, se deberá de entregar el $100\%$ de los ejercicios resueltos. En caso contrario, sólo se revisarán los ejercicios, pero no se tomará en cuenta como parte de la calificación por tareas. Un examen-tarea se considera acreditado cuando la calificación obtenida es mayor o igual a seis. En caso de que en alguno (o más) examen-tarea, la calificación sea menor a seis, ya se es candidato a presentar el examen final del curso.
\item \textbf{Proyecto final $\mathbf{40\%}$} : A la mitad del semestre se presentará una lista con varios proyectos de los cuales se deberá elegir uno, para desarrollarlo conforme se continúa con la revisión del contenido del curso, para entregarlo de manera completa antes de que concluya el semestre.
\end{enumerate}
La calificación final del curso se obtendrá de las calificaciones de cada uno de los componentes de la evaluación: ejercicios en clase, de los examen-tareas y del proyecto final. En el caso de obtener una calificación mayor o igual a $6$, será la que se asentará en el acta del curso.

\section{Herramienta de trabajo.}

Es necesario contar con una herramienta de trabajo que nos permita generar un algoritmo computacional con el que podamos comprobar la solución y consistencia de nuestro problema, para ello, utilizaremos:
\begin{enumerate}[label=\roman*)]
\item Python como lenguaje de programación. Se tendrá un Tema 0 a modo de Introducción Breve a Python.
\item Jupyter como cuadernos de trabajo (notebooks)
\item Para facilitar la distribución de los notebooks iniciales, se tendrá disponible un repositorio en GitHub.
\end{enumerate}

\section{Examen final.}

Para presentar el examen final del curso se deben de cumplir cada una de las siguientes condiciones:
\begin{enumerate}\label{ref:criterios_final}
\item Que en un examen-tarea (o más), la calificación sea menor a seis. Si los examen-tarea tienen calificación aprobatoria, no se permite presentar el examen final para \enquote{subir} la calificación del curso.
\item Se hayan entregado los tres examen-tarea parciales.
\item Se haya entregado el proyecto final.
\end{enumerate}
En caso de que no se cumplan las condiciones anteriores, no se podrá presentar el examen final. De acuerdo al Reglamento de Estudios Profesionales, habrá dos oportunidades para presentar el examen final, cuyas fechas se indican en el calendario del semestre 2023-1.
\par
Puntalizando sobre el examen final:
\begin{enumerate}[label=\roman*)]
\item Si en la primera ronda de examen final, la calificación obtenida es aprobatoria (mayor o igual a seis), ésta es la que se asentará en el acta del curso, ya no se promedia con los otros elementos de evaluación.
\item Si la calificación del examen final en la primera ronda es no aprobatoria, se aplicará nuevamente un examen final en la segunda ronda. La calificación obtenida en esta segunda ronda, es la que se asentará en el acta del curso.
\item Si el alumno no se presenta a la primera ronda del examen final, tendrá cinco como calificación final. Ya no podrá presentar la segunda ronda del examen final.
\end{enumerate}
\par
\textbf{Importante: } \emph{En caso de haber presentado al menos un examen-tarea y/o haber entregado al menos un ejercicio}, pero si ya no se tiene un posterior registro de entregas, se considera que abandonaron el curso, al no cumplir con los puntos de la lista del numeral \ref{ref:criterios_final}, no se podrá presentar el examen final del curso.
\par
Se asentará en el acta de calificaciones \textcolor{blue}{No Presentó (NP)}, si y solo si: el alumno no entrega ejercicio alguno y no entrega algún examen-tarea (¿?). Ocupando nuevamente el Reglamento de Estudios Profesionales, tomen en cuenta que:
\begin{itemize}
\setlength\itemsep{1pt}
\item No \enquote{se guardan calificaciones}.
\item No se renuncia a una calificación.
\end{itemize}

\section{Fechas importantes.}

\begin{itemize}
\setlength\itemsep{1pt}
\item Lunes 15 de agosto. Inicio del semestre 2023-1.
\item \textcolor{red}{Jueves 15 de septiembre. Día feriado.}
\item \textcolor{red}{Martes 1 de noviembre. Día feriado.}
\item Viernes 2 de diciembre. Fin de semestre 2023-1.
\item Del lunes 5 al viernes 9 de diciembre, primera semana de finales.
\item Del martes 13 al viernes 16 de diciembre, segunda semana de finales.
\end{itemize}

\section{Bibliografía.}

Se recomienda la consulta de los siguientes textos, en cada uno de los temas se propocionará bibliografía adicional para una mejor comprensión del tema.
\nocite{*}
\printbibliography[keyword={computacional}, title={Referencias Física Computacional}]
\printbibliography[keyword={python}, title={Referencias pyhton, matplolib, jupyter}]

\end{document}