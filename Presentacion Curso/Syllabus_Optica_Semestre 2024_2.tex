\documentclass[12pt]{article}
\usepackage[left=0.25cm,top=1cm,right=0.25cm,bottom=1cm]{geometry}
%\usepackage[landscape]{geometry}
\textwidth = 20cm
\hoffset = -1cm
\usepackage[utf8]{inputenc}
\usepackage[spanish,es-tabla, es-lcroman]{babel}
\usepackage[autostyle,spanish=mexican]{csquotes}
\usepackage[tbtags]{amsmath}
\usepackage{nccmath}
\usepackage{amsthm}
\usepackage{amssymb}
\usepackage{mathrsfs}
\usepackage{graphicx}
\usepackage{subfig}
\usepackage{caption}
%\usepackage{subcaption}
\usepackage{standalone}
\usepackage[outdir=./Imagenes/]{epstopdf}
\usepackage{siunitx}
\usepackage{physics}
\usepackage{color}
\usepackage{float}
\usepackage{hyperref}
\usepackage{multicol}
\usepackage{multirow}
%\usepackage{milista}
\usepackage{anyfontsize}
\usepackage{anysize}
%\usepackage{enumerate}
\usepackage[shortlabels]{enumitem}
\usepackage{capt-of}
\usepackage{bm}
\usepackage{mdframed}
\usepackage{relsize}
\usepackage{placeins}
\usepackage{empheq}
\usepackage{cancel}
\usepackage{pdfpages}
\usepackage{wrapfig}
\usepackage[flushleft]{threeparttable}
\usepackage{makecell}
\usepackage{fancyhdr}
\usepackage{tikz}
\usepackage{bigints}
\usepackage{menukeys}
\usepackage{tcolorbox}
\tcbuselibrary{breakable}
\usepackage{scalerel}
\usepackage{pgfplots}
\usepackage{pdflscape}
\pgfplotsset{compat=1.16}
\spanishdecimal{.}
\renewcommand{\baselinestretch}{1.5} 
\renewcommand\labelenumii{\theenumi.{\arabic{enumii}})}

\newcommand{\python}{\texttt{python}}
\newcommand{\textoazul}[1]{\textcolor{blue}{#1}}
\newcommand{\azulfuerte}[1]{\textcolor{blue}{\textbf{#1}}}
\newcommand{\funcionazul}[1]{\textcolor{blue}{\textbf{\texttt{#1}}}}

\newcommand{\pderivada}[1]{\ensuremath{{#1}^{\prime}}}
\newcommand{\sderivada}[1]{\ensuremath{{#1}^{\prime \prime}}}
\newcommand{\tderivada}[1]{\ensuremath{{#1}^{\prime \prime \prime}}}
\newcommand{\nderivada}[2]{\ensuremath{{#1}^{(#2)}}}


\newtheorem{defi}{{\it Definición}}[section]
\newtheorem{teo}{{\it Teorema}}[section]
\newtheorem{ejemplo}{{\it Ejemplo}}[section]
\newtheorem{propiedad}{{\it Propiedad}}[section]
\newtheorem{lema}{{\it Lema}}[section]
\newtheorem{cor}{Corolario}
\newtheorem{ejer}{Ejercicio}[section]

\newlist{milista}{enumerate}{2}
\setlist[milista,1]{label=\arabic*)}
\setlist[milista,2]{label=\arabic{milistai}.\arabic*)}
\newlength{\depthofsumsign}
\setlength{\depthofsumsign}{\depthof{$\sum$}}
\newcommand{\nsum}[1][1.4]{% only for \displaystyle
    \mathop{%
        \raisebox
            {-#1\depthofsumsign+1\depthofsumsign}
            {\scalebox
                {#1}
                {$\displaystyle\sum$}%
            }
    }
}
\def\scaleint#1{\vcenter{\hbox{\scaleto[3ex]{\displaystyle\int}{#1}}}}
\def\scaleoint#1{\vcenter{\hbox{\scaleto[3ex]{\displaystyle\oint}{#1}}}}
\def\scaleiiint#1{\vcenter{\hbox{\scaleto[3ex]{\displaystyle\iiint}{#1}}}}
\def\bs{\mkern-12mu}

\newcommand{\Cancel}[2][black]{{\color{#1}\cancel{\color{black}#2}}}


\usepackage[sfdefault]{roboto}  %% Option 'sfdefault' only if the base font of the document is to be sans serif
\usepackage[T1]{fontenc}

\author{M. en C. Gustavo Contreras Mayén. \texttt{gux7avo@ciencias.unam.mx}\\
M. en C. Abraham Lima Buendía. \texttt{abraham3081@ciencias.unam.mx}}
\title{\vspace*{-4cm}Syllabus del Curso de Óptica \\ {\large Semestre 2024-2 Grupo 8365}}
\date{ }

% \makeatletter
% \renewcommand{\@biblabel}[1]{}
% \renewenvironment{thebibliography}[1]
%      {\section*{\refname}%
%       \@mkboth{\MakeUppercase\refname}{\MakeUppercase\refname}%
%       \list{}%
%            {\labelwidth=0pt
%             \labelsep=0pt
%             \leftmargin1.5em
%             \itemindent=-1.5em
%             \advance\leftmargin\labelsep
%             \@openbib@code
%             }%
%       \sloppy
%       \clubpenalty4000
%       \@clubpenalty \clubpenalty
%       \widowpenalty4000%
%       \sfcode`\.\@m}
% \makeatother

\usepackage[backend=biber, style=ieee, sorting=ynt]{biblatex}
% \addbibresource{LibrosFC.bib}


\begin{document}

\renewcommand*{\theenumi}{\thesection.\arabic{enumi}}
\renewcommand*{\theenumii}{\theenumi.\arabic{enumii}}

\maketitle
\fontsize{12}{12}\selectfont

\textbf{Lugar:} Salón P109.
\par
\textbf{Horario:} Lunes, Miércoles y Viernes de 18 a 20 pm.

\section{Objetivos.}

\subsection{Objetivos.}

\begin{enumerate}
\item Enseñar los fundamentos de la óptica geométrica y la óptica física, así como sus aplicaciones.
\item El alumno reconocerá que los diversos fenómenos ópticos observados en la naturaleza, así como sus aplicaciones son una consecuencia directa de la teoría electromagnética revisada en los cursos previos.
\item El alumno resolverá de la mejor manera los problemas modelo del curso.
\end{enumerate}

\section*{Temario.}

Se trabajará el temario oficial de la asignatura, que está disponible en:

\href{https://www.fciencias.unam.mx/sites/default/files/temario/584.pdf}{https://www.fciencias.unam.mx/sites/default/files/temario/584.pdf}

\section{Tema 1: Ondas electromagnéticas.}
\begin{enumerate}
\item Conceptos básicos y propiedades de las ondas.
\item La ecuación de onda, solución general y principio de superposición.
\item Principio de Huygens, rayos y superficies de onda.
\item Adición de ondas: a) de la misma frecuencia, b) frecuencia casi idéntica. Conceptos de velocidad de fase y grupo.
\item Teorema de Fourier.
\end{enumerate}

\section{Tema 2: Fundamentos de electromagnetismo.}
\begin{enumerate}
\item Ecuaciones de Maxwell en a) el vacío y b) medios materiales. Naturaleza electromagnética de
la luz.
\item Energía en el campo electromagnético, teorema de Poynting.
\item Polarización, Ley de Malus y Vectores de Jones.
\end{enumerate}

\section{Tema 3: Ecuaciones de Fresnel.}
\begin{enumerate}
\item Ecuaciones de Maxwell en términos de las condiciones de frontera.
\item Reflexión y refracción de ondas electromagnéticas en medios dieléctricos isotrópicos.
\item Las ecuaciones de Fresnel. Coeficientes de amplitud e intensidad.
\item Ángulo de Brewster, cambios de fase, reflexión total interna frustrada.
\end{enumerate}

\section{Tema 4: Teoría de la dispersión}
\begin{enumerate}
\item Propagación de la luz en medios dieléctricos isotrópicos.
\item Dispersión normal y anómala. Absorción.
\item Propagación de ondas electromagnéticas en medios a) dieléctricos y b) conductores. Frecuencia de plasma.
\end{enumerate}

\section{Tema 5: Fundamentos de la óptica geométrica.}
\begin{enumerate}
\item Límites de aplicabilidad de la óptica geométrica.
\item Camino óptico. Principio de Fermat.
\item Leyes de la óptica geométrica: reflexión y refracción en superficies planas y curvas.
\item Reflexión y refracción en superficies esféricas.
\item Lentes delgadas y espejos, aproximación paraxial, ecuación de Gauss y fórmula del fabricante de lentes.
\item Formación de imágenes y Amplificación a) transversal y b) longitudinal.
\item Sistemas ópticos: Ojo humano, microscopio, telescopio, cámara fotográfica, número F.
\item Prismas, diferentes tipos y aplicaciones.
\item Aberraciones.
\end{enumerate}

\section{Tema 6: Interferencia.}
\begin{enumerate}
\item Definiciones y conceptos preliminares.
\item Condiciones para observar interferencia. Leyes de Fresnel-Arago.
\item Interferencia por división de frente de onda.
\item Interferencia por división de amplitud.
\item Tipo y localización de franjas.
\item Interferómetros y sus aplicaciones.
\item Películas delgadas. Aplicaciones.
\end{enumerate}

\section{Tema 7: Difracción.}
\begin{enumerate}
\item Introducción. Principio de Huygens-Fresnel.
\item Obstáculos. Principio de Babinet.
\item Difracción de Fraunhoffer.
\item Difracción de Fresnel. Espiral de Cornu.
\item Rejillas de difracción. Aplicaciones.
\end{enumerate}

\section{Tema 8: Óptica de cristales}
\begin{enumerate}
\item Propagación de la luz en medios cristalinos.
\item Superficie número de ondas y superficie índice de refracción (descripción).
\item Birrefringencia, dicroísmo, retardadores, compensadores y polarizadores.
\item Actividad óptica.
\item Efectos ópticos inducidos. (Faraday, Kerr, Pockels, foto-elasticidad).
\end{enumerate}

\section{Evaluación del curso.}

Los elementos y la proporción de la calificación total del curso, se distribuyen de la siguiente manera:

\begin{enumerate}
\item \textbf{Tareas $\mathbf{40\%}$} : Se tendrán cuatro tareas durante el curso, se les proporcionará de manera adelantada y con fecha de entrega definida, no se recibirán entregas extemporáneas.

Para que la tarea se considere, se deberá de entregar el $60\%$ de los ejercicios resueltos. En caso contrario, sólo se revisarán los ejercicios, pero no se tomará en cuenta como parte de la calificación por tareas.
\item \textbf{Exámenes $\mathbf{60\%}$} : Habrá tres exámenes en clase. 
\item Se indicará oportunamente el día del examen y los temas correspondientes, que se resolverán y entregarán durante la clase.
\end{enumerate}

\section{Metodología de Enseñanza.}

\noindent
\textbf{Antes de la clase.}
\\
Para facilitar la discusión en el aula, el alumno revisará el material de trabajo que se le proporcionará oportunamente, así como la solución de algunos ejercicios, de tal manera que llegará a la clase conociendo el tema a desarrollar. Daremos por entendido de que el alumno realizará la lectura y actividades establecidas.
\\
\textbf{Durante la clase.}
\\
Se dará un tiempo para la exposición con diálogo y discusión del material de trabajo con los temas a cubrir durante el semestre.
\\
\textbf{Después de la clase.}
\\
Al concluir la clase, se tendrán ejercicios a resolver, para que pueda repasar el tema visto en clase.

\subsection{Plataforma Moodle.}

En este semestre el curso se impartirá en modalidad presencial, se mantendrá el uso de la plataforma Moodle para el mismo, en donde se incluirán materiales de consulta: ejercicios adicionales, lecturas, artículos, enlaces a videos, a archivos para utilizar, por lo que se tendrá un apoyo adicional, de manera que contarán con herramientas y materiales complementarios.

\section{Examen final.}

Para presentar el examen final del curso se deben de cumplir cada una de las siguientes condiciones:
\begin{enumerate}\label{ref:criterios_final}
\item Que en un examen (o más), la calificación sea menor a seis. Si los examen tienen calificación aprobatoria, no se permite presentar el examen final para \enquote{subir} la calificación del curso.
\item Se hayan entregado los tres exámenes parciales.
\end{enumerate}
En caso de que no se cumplan las condiciones anteriores, no se podrá presentar el examen final. De acuerdo al Reglamento de Estudios Profesionales, habrá dos oportunidades para presentar el examen final, cuyas fechas se indican en el calendario del semestre 2024-2.
\par
Puntalizando sobre el examen final:
\begin{enumerate}[label=\roman*)]
\item Si en la primera ronda de examen final, la calificación obtenida es aprobatoria (mayor o igual a seis), ésta es la que se asentará en el acta del curso, ya no se promedia con los otros elementos de evaluación.
\item Si la calificación del examen final en la primera ronda es no aprobatoria, se aplicará nuevamente un examen final en la segunda ronda. La calificación obtenida en esta segunda ronda, es la que se asentará en el acta del curso.
\item Si el alumno no se presenta a la primera ronda del examen final, tendrá cinco como calificación final. Ya no podrá presentar la segunda ronda del examen final.
\end{enumerate}
\par
\textbf{Importante: } \emph{En caso de haber presentado al menos un examen y/o haber entregado al menos un ejercicio}, pero si ya no se tiene un posterior registro de entregas, se considera que abandonaron el curso, al no cumplir con los puntos de la lista del numeral \ref{ref:criterios_final}, no se podrá presentar el examen final del curso.
\par
Se asentará en el acta de calificaciones \textcolor{blue}{No Presentó (NP)}, si y solo si: el alumno no entrega ejercicio alguno y no entrega algún examen-tarea (¿?). Ocupando nuevamente el Reglamento de Estudios Profesionales, tomen en cuenta que:
\begin{itemize}
\setlength\itemsep{1pt}
\item No \enquote{se guardan calificaciones}.
\item No se renuncia a una calificación.
\end{itemize}

\section{Fechas importantes.}

\begin{itemize}
\setlength\itemsep{1pt}
\item Lunes 29 de enero de 2024. Inicio del semestre 2024-2.
\item \textcolor{red}{Lunes 5 de febrero. Día feriado.}
\item \textcolor{red}{Lunes 18 de marzo. Día feriado.}
\item \textcolor{red}{Lunes 25 al 29 de marzo. Días feriados.}
\item \textcolor{red}{Miércoles 1 de mayo. Día feriado.}
\item \textcolor{red}{Viernes 10 de mayo. Día feriado.}
\item \textcolor{red}{Miércoles 15 de mayo. Día feriado.}
\item \textcolor{red}{Martes 1 de noviembre. Día feriado.}
\item Viernes 24 de mayo. Fin de semestre 2024-2.
\item Del lunes 27 al viernes 31 de mayo, primera semana de finales.
\item Del martes 3 al viernes 7 de junio, segunda semana de finales.
\end{itemize}

\section{Bibliografía.}

Se recomienda la consulta de los siguientes textos, en cada uno de los temas se propocionará bibliografía adicional para una mejor comprensión del tema.
% \nocite{*}
% \printbibliography[keyword={computacional}, title={Referencias Física Computacional}]
% \printbibliography[keyword={python}, title={Referencias pyhton, matplolib, jupyter}]

\end{document}