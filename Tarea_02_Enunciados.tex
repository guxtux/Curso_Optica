\documentclass[14pt]{extarticle}
\usepackage[utf8]{inputenc}
\usepackage[T1]{fontenc}
\usepackage[spanish,es-lcroman]{babel}
\usepackage{amsmath}
\usepackage{amsthm}
\usepackage{physics}
\usepackage{tikz}
\usepackage{float}
\usepackage{calc}
\usepackage[autostyle,spanish=mexican]{csquotes}
\usepackage[per-mode=symbol]{siunitx}
\usepackage{gensymb}
\usepackage{multicol}
\usepackage{enumitem}
\usepackage{setspace}
\usepackage[left=2.00cm, right=2.00cm, top=2.00cm, 
     bottom=2.00cm]{geometry}
\usepackage{Estilos/ColoresLatex}
\usepackage{makecell}

\newcommand{\textocolor}[2]{\textbf{\textcolor{#1}{#2}}}
\sisetup{per-mode=symbol}
\decimalpoint
\sisetup{bracket-numbers = false}
\newlength{\depthofsumsign}
\setlength{\depthofsumsign}{\depthof{$\sum$}}
\newcommand{\nsum}[1][1.4]{% only for \displaystyle
    \mathop{%
        \raisebox
            {-#1\depthofsumsign+1\depthofsumsign}
            {\scalebox
                {#1}
                {$\displaystyle\sum$}%
            }
    }
}


\author{\normalsize{M. en C. Gustavo Contreras Mayén.} \quad \normalsize{\texttt{gux7avo@ciencias.unam.mx}} \\
\normalsize{M. en C. Abraham Lima Buendía.} \quad \normalsize{\texttt{abraham3081@ciencias.unam.mx}}}
\title{\vspace*{-2cm} Tarea 2 (Primera parte) - Curso de Óptica}
\date{ }

\begin{document}

\setstretch{1.3}

\maketitle
\fontsize{14}{14}\selectfont

\begin{itemize}
\item \textbf{Problema 1: }Usando el pricipio de Fermat deduzca la ley de Snell para una superficie plana, su demostración debe mostrar que el rayo incidente, el reflejado y el transmitido están definidos en el mismo plano.
\item \textbf{Problema 2}:Imagine que tiene una placa de vidrio no absorbente de índice de refracción $n$ y grosor $\Delta y$, que se encuentra entre un foco $S$ y un observador $P$.

a) Si la onda no está obstruida (sin que la placa esté presente), demuestra que:
\begin{align*}
E_{u} = E_{0} \, \exp \left[i \omega \left(t - \dfrac{y}{c} \right) \right]
\end{align*}

b) Demuestre que con la placa en su lugar el observador ve una onda:
\begin{align*}
E_{p} = E_{0} \, \exp \left[ \displaystyle i \omega \left( t - \left\{ n-1\right\} \dfrac{\Delta y}{c}\dfrac{y}{c}\right) \right]
\end{align*}
que se puede aproximar como:
\begin{align*}
E_{p} = E_{u} + \dfrac{\omega (n - 1) \Delta y}{c} \, E_{u} \exp\left[-i \dfrac{\pi}{2} \right]
\end{align*}
\item \textbf{Problema 3}: La luz incide sobre una interfaz aire vidrio. Si el índice de refracción del vidrio es $1.70$ averigüe el ángulo incidente $\theta_{i}$ tal que el ángulo de transmisión sea igual a $\dfrac{\theta_{i}}{2}$
\item \textbf{Problema 4}: Suponga que se enfoca una cámara sobre una letra impresa en una hoja. Cubra la imagen con un porta objetos de \SI{1}{\milli\meter} de grosor e índice de refracción $n = 1.55$ ¿Cuánto deberá alejarse la cámara en el aire para que la letra siga enfocada?
\end{itemize}

\end{document}
