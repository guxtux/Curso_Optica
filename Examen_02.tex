\documentclass[14pt]{extarticle}
\usepackage[utf8]{inputenc}
\usepackage[T1]{fontenc}
\usepackage[spanish,es-lcroman]{babel}
\usepackage{amsmath}
\usepackage{amsthm}
\usepackage{physics}
\usepackage{tikz}
\usepackage{float}
\usepackage{calc}
\usepackage[autostyle,spanish=mexican]{csquotes}
\usepackage[per-mode=symbol]{siunitx}
\usepackage{gensymb}
\usepackage{multicol}
\usepackage{enumitem}
\usepackage{setspace}
\usepackage[left=2.00cm, right=2.00cm, top=2.00cm, 
     bottom=2.00cm]{geometry}
\usepackage{Estilos/ColoresLatex}
\usepackage{makecell}

\newcommand{\textocolor}[2]{\textbf{\textcolor{#1}{#2}}}
\sisetup{per-mode=symbol}
\decimalpoint
\sisetup{bracket-numbers = false}
\newlength{\depthofsumsign}
\setlength{\depthofsumsign}{\depthof{$\sum$}}
\newcommand{\nsum}[1][1.4]{% only for \displaystyle
    \mathop{%
        \raisebox
            {-#1\depthofsumsign+1\depthofsumsign}
            {\scalebox
                {#1}
                {$\displaystyle\sum$}%
            }
    }
}


\author{\normalsize{M. en C. Gustavo Contreras Mayén.} \quad \normalsize{\texttt{gux7avo@ciencias.unam.mx}} \\
\normalsize{M. en C. Abraham Lima Buendía.} \quad \normalsize{\texttt{abraham3081@ciencias.unam.mx}}}
\title{\vspace*{-2cm} Examen Parcial 2 - Curso de Óptica}
\date{ }

\begin{document}

\maketitle
\fontsize{14}{14}\selectfont

\setstretch{1.3}

\begin{itemize}
\item \textbf{Problema 1:} La imagen proyectada por una lente equiconvexa $n = 1.50$ de una rana con una  altura de \SI{5}{\centi\meter} y que se halla a \SI{0.6}{\centi\meter} de una pantalla tienen que medir \SI{25}{\centi\meter} de alto. Calcula el radio de la lente.
\item \textbf{Problema 2:} Demuestra que la separación mínima entre los puntos conjugados objeto e imagen en una lente delgada es $4 \, f$.
\item \textbf{Problema 3:} Considera una lente positiva delgada $L1$, demuestra que si una segunda lente se coloca a la distancia focal de $L1$ el aumento de la imagen no cambia. Recurre a las fórmulas vistas en clase y explica su resultado con un diagrama de rayos.
\item \textbf{Problema 4:} Diseña el ojo de un robot con un espejo esférico cóncavo de tal manera que la imagen de un objeto de altura \SI{1}{\meter} y colocado a una distancia de \SI{10}{\meter} llene su fotodetector cuadrado de \SI{1}{\centi\meter}. ¿Dónde debería estar ubicado dicho detector con respecto al espejo? ¿Cuál debería ser la distancia focal del espejo? Traza un diagrama de rayos.
\item \textbf{Problema 5:} Determina la apertura numérica de una única fibra óptica con recubrimiento sabiendo que el índice del núcleo y el revestimiento es de \num{1.62} y \num{1.52} respectivamente. ¿Cuál es el máximo ángulo de aceptación cuando se sumerge en el aire? ¿Qué le pasaría a un rayo incidente a \ang{45}?

\newpage

\item \textbf{Problema 6:} Considerando la operación de un espejo esférico, demuestra que las posiciones del objeto y la imagen son proporcionadas por:
\begin{align*}
s_{0} &= \dfrac{f \, \left( M_{T} - 1 \right)}{M_{T}} \\[0.5em]
s_{I} &= - f \left(M_{T} - 1 \right)
\end{align*}
\end{itemize}

\end{document}