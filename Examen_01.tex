\documentclass[14pt]{extarticle}
\usepackage[utf8]{inputenc}
\usepackage[T1]{fontenc}
\usepackage[spanish,es-lcroman]{babel}
\usepackage{amsmath}
\usepackage{amsthm}
\usepackage{physics}
\usepackage{tikz}
\usepackage{float}
\usepackage{calc}
\usepackage[autostyle,spanish=mexican]{csquotes}
\usepackage[per-mode=symbol]{siunitx}
\usepackage{gensymb}
\usepackage{multicol}
\usepackage{enumitem}
\usepackage{setspace}
\usepackage[left=2.00cm, right=2.00cm, top=2.00cm, 
     bottom=2.00cm]{geometry}
\usepackage{Estilos/ColoresLatex}
\usepackage{makecell}

\newcommand{\textocolor}[2]{\textbf{\textcolor{#1}{#2}}}
\sisetup{per-mode=symbol}
\decimalpoint
\sisetup{bracket-numbers = false}
\newlength{\depthofsumsign}
\setlength{\depthofsumsign}{\depthof{$\sum$}}
\newcommand{\nsum}[1][1.4]{% only for \displaystyle
    \mathop{%
        \raisebox
            {-#1\depthofsumsign+1\depthofsumsign}
            {\scalebox
                {#1}
                {$\displaystyle\sum$}%
            }
    }
}


\author{\normalsize{M. en C. Gustavo Contreras Mayén.} \quad \normalsize{\texttt{gux7avo@ciencias.unam.mx}} \\
\normalsize{M. en C. Abraham Lima Buendía.} \quad \normalsize{\texttt{abraham3081@ciencias.unam.mx}}}
\title{\vspace*{-2cm} Examen Parcial 1 - Curso de Óptica}
\date{ }

\begin{document}

\maketitle
\begin{enumerate}
\item Confirma que la matriz :
\begin{align*}
\begin{bmatrix}
1 & 0 & 0 & 0 \\
0 & 0 & 0 & -1 \\
0 & 0 & 1 & 0 \\
0 & 1 & 0 & 0        
\end{bmatrix}
\end{align*}
actuará como matriz de Muller, para una lámina de un cuarto de onda cuyo eje rápido se halla a +\ang{45} Ilumina con un haz de luz polarizada lineal a \ang{45} ¿Qué pasa? ¿Qué emerge al entrar en el dispositivo un estado $P$ horizontal? 
\item Escribe una expresión para una onda luminosa de estado $P$ con frecuencia angular $\omega$ y amplitud $E_{0}$ propagándose a lo largo de una línea en el plano $x-y$ a \ang{45} respecto al eje $x$, cuyo plano de vibración está en el plano $x-y$. En $t = 0$ y $ x = 0$ el campo es cero.
\item Demuestra que la resultante de dos ondas:
\begin{align*}
E_{1} &= E_{01} \sin \left[ \omega t - k \left( x + \Delta x \right) \right] \\[0.5em]
E_{2} &= E_{01} \sin \left[ \omega t - k \, x \right]
\end{align*}
es la siguiente:
\begin{align*}
E = 2\, E_{01} \, \cos \left[ \dfrac{k \, \Delta \, x}{2} \right] \, \sin \left[ \omega t - k \left( x + \Delta \, x \right) \right]
\end{align*}
\item El campo eléctrico de una onda electromagnética que viaja en la dirección positiva de $x$ está dada por:
\begin{align*}
\va{E} = \vu{j} \, \mathrm{ \sin \left[ \mathit{\dfrac{\pi \, z}{z_{0}}} \right] \, \cos \left[ \mathit{ k x - \omega t} \right]}
\end{align*}
determina una expresión para $k$ y calcula la velocidad de fase de la onda. 
\end{enumerate}

\end{document}
