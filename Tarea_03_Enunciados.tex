\documentclass[14pt]{extarticle}
\usepackage[utf8]{inputenc}
\usepackage[T1]{fontenc}
\usepackage[spanish,es-lcroman]{babel}
\usepackage{amsmath}
\usepackage{amsthm}
\usepackage{physics}
\usepackage{tikz}
\usepackage{float}
\usepackage{calc}
\usepackage[autostyle,spanish=mexican]{csquotes}
\usepackage[per-mode=symbol]{siunitx}
\usepackage{gensymb}
\usepackage{multicol}
\usepackage{enumitem}
\usepackage{setspace}
\usepackage[left=2.00cm, right=2.00cm, top=2.00cm, 
     bottom=2.00cm]{geometry}
\usepackage{Estilos/ColoresLatex}
\usepackage{makecell}

\newcommand{\textocolor}[2]{\textbf{\textcolor{#1}{#2}}}
\sisetup{per-mode=symbol}
\decimalpoint
\sisetup{bracket-numbers = false}
\newlength{\depthofsumsign}
\setlength{\depthofsumsign}{\depthof{$\sum$}}
\newcommand{\nsum}[1][1.4]{% only for \displaystyle
    \mathop{%
        \raisebox
            {-#1\depthofsumsign+1\depthofsumsign}
            {\scalebox
                {#1}
                {$\displaystyle\sum$}%
            }
    }
}


\author{\normalsize{M. en C. Gustavo Contreras Mayén.} \quad \normalsize{\texttt{gux7avo@ciencias.unam.mx}} \\
\normalsize{M. en C. Abraham Lima Buendía.} \quad \normalsize{\texttt{abraham3081@ciencias.unam.mx}}}
\title{\vspace*{-2cm} Tarea 2 (Segunda parte) - Curso de Óptica}
\date{ }

\begin{document}

\maketitle
\fontsize{14}{14}\selectfont

\setstretch{1.3}

\begin{itemize}
\item \textbf{Problema 1:} Demuestra que en la aproximación paraxial, la relación objeto imagen en un espejo esférico, está determinada por:
\begin{align*}
s_{0}^{-1} + s_{I}^{-1} = 2 \, R^{-1}
\end{align*}
siendo $s_{0}$ la distancia del vértice al punto objeto, $s_{I}$ la distancia de la imagen al vértice del espejo y $R$ el radio de la superficie.
\item \textbf{Problema 2:} Una vela de \SI{6}{\centi\meter} de altura se localiza a \SI{10}{\centi\meter} de una lente cóncava cuya distancia focal es de $-30cm$. Determina la posición de la imagen y descríbela en detalle. Dibuja un diagrama de rayos apropiado.
\item \textbf{Problema 3:} Un hombre cuyo rostro dista \SI{25}{\centi\meter} de una cuchara sopera que se mira en el cuenco de la misma, ve su imagen con un aumento de $-0.064$. Determina el radio de la cuchara.
\item \textbf{Problema 4:} Traza un diagrama de rayos para la combinación de dos lentes positivas para una separación mayor a la distancia focal entre las lentes. Discute e ilustra el diagrama de rayos, discute el límite en el que la separación es igual a la distancia focal.
\end{itemize}


\end{document}