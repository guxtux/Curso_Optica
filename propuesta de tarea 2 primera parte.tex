%% LyX 2.3.7 created this file.  For more info, see http://www.lyx.org/.
%% Do not edit unless you really know what you are doing.
\documentclass[english]{article}
\usepackage[T1]{fontenc}
\usepackage[latin9]{inputenc}
\usepackage{units}
\usepackage{babel}
\begin{document}
\title{Tarea 2 �ptica (parte 1)}
\maketitle
\begin{itemize}
\item \textbf{Problema 1: }Usando el pricipio de Fermant deduzca la ley
de Snell para una superficie plana, su demostraci�n debe mostrar que
el rayo incidente, el reflejado y el transmitido est�n definidos en
el mismo plano.
\item \textbf{Problema 2}:Imagine que tiene una placa de vidrio no absorbente
de �ndice de refracci�n $n$ y grosor $\Delta y$, que se encuentra
entre un foco $S$ y un observador $P$. a) Sil a onda n obstruida
(sin que la placa est� presente) es $E_{u}=E_{0}\exp\left[i\omega\left(t-\nicefrac{y}{c}\right)\right]$,
demuestre que con la placa en su lugar el observador ve una onda $E_{p}=E_{0}\exp\left[i\omega\left(t-\left\{ n-1\right\} \nicefrac{\Delta y}{c}\nicefrac{y}{c}\right)\right]$
b) $E_{p}=E_{u}+\frac{\omega(n-1)\Delta y}{c}E_{u}\exp\left[-i\nicefrac{\pi}{2}\right]$.
\item \textbf{Problema 3}: La luz incide sobre una interfaz aire vidrio.
Si el �ndice de refracci�n del vidrio es 1.70 averig�e el �ngulo incidente
$\theta_{i}$ tal que el �ngulo de transmisi�n sea igual a $\nicefrac{\theta_{i}}{2}$
\item \textbf{Problema 4}: Suponga que se enfoca una c�mara sobre una letra
impresa en una hoja. Cubra la imagen con un porta objetos de un $mm$
de grosor e �ndice de refracci�n $n=1.55$�cu�nto deber� alejarse
la c�mara en el aire para que la letra siga enfocada ?
\end{itemize}

\end{document}
