%% LyX 2.3.7 created this file.  For more info, see http://www.lyx.org/.
%% Do not edit unless you really know what you are doing.
\documentclass[english]{article}
\usepackage[T1]{fontenc}
\usepackage[latin9]{inputenc}
\usepackage{units}
\usepackage{stackrel}

\makeatletter
%%%%%%%%%%%%%%%%%%%%%%%%%%%%%% Textclass specific LaTeX commands.
\newenvironment{lyxlist}[1]
	{\begin{list}{}
		{\settowidth{\labelwidth}{#1}
		 \setlength{\leftmargin}{\labelwidth}
		 \addtolength{\leftmargin}{\labelsep}
		 \renewcommand{\makelabel}[1]{##1\hfil}}}
	{\end{list}}

\makeatother

\usepackage{babel}
\begin{document}
\title{Tarea 1 �ptica}
\maketitle
\begin{center}
{\large{}02/24/24}{\large\par}
\par\end{center}
\begin{lyxlist}{00.00.0000}
\item [{\textbf{Problema}}] \textbf{1}: Demostrar que el momento can�nico
del campo electromagn�tico en el vac�o es: $\vec{p}=\epsilon_{0}\mu_{0}\vec{S}$
\item [{\textbf{Problema}}] \textbf{2}: Escriba las ecuaciones de Maxwell
en su forma covariante.
\item [{\textbf{Problema}}] \textbf{3}:Bajo las consideraciones $0\leq x\leq a$,
$H_{0},\,a,\,k,\,y\,\omega$ son constantes y se satisface la relaci�n
de dispersi�n $\mu_{0}\epsilon_{0}\omega^{2}=k^{2}+\left(\nicefrac{\pi}{a}\right)^{2}$.
a) Muestre que las siguientes expresiones satisfacen las ecuaciones
de maxwell en el vac�o: $E_{y}=-H_{0}\mu_{0}\omega\left(\frac{a}{\pi}\right)sen\left(kz-\omega t\right)sen\left(\frac{\pi x}{a}\right)$,
$H_{x}=H_{0}k\left(\frac{a}{\pi}\right)sen\left(kz-\omega t\right)sen\left(\frac{\pi x}{a}\right)$,
$H_{z}=H_{0}cos\left(kz-\omega t\right)cos\left(\frac{\pi x}{a}\right)$,
el resto de las componentes son cero y puede asumir que no hay presencia
de cargas o corrientes libres. b) Encontrar la corriente de desplazamiento.
c) Encontrar el vector de Poynting.
\item [{\textbf{Problema}}] \textbf{4}:Una onda estacionaria viene dada
por $E=100sen\left(\frac{2}{3}\pi x\right)cos\left(5\pi t\right)$,
determine dos ondas que pueden superponese para crearla.
\item [{\textbf{Problema}}] \textbf{5}:Demuestre que en la representaci�n
de Fourier la funci�n $f(x)=|sen(x)|$ es: $f(x)=\frac{2}{\pi}-\frac{4}{\pi}\stackrel[m=1]{\infty}{\sum}\frac{cos(2mx)}{4m^{2}-1}$
\end{lyxlist}

\end{document}
